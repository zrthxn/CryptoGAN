\documentclass[12pt]{article}

\usepackage{a4}
\usepackage{lipsum}
\usepackage{graphicx}
\usepackage{algorithmic}
\usepackage{cite}
\usepackage{xcolor}
\usepackage{amsmath,amssymb,amsfonts}
\usepackage{textcomp}

\title{A Review on Adversarial Neural Cryptography}

\author{
  {\bf Alisamar Husain}\\
  Dept. of Electrical Engineering\\
  Jamia Millia Islamia
}

\date{}

\begin{document}
  \maketitle

  \begin{abstract}
    Artificial neural networks are well known for their ability to selectively explore the solution
    space of a given problem. This feature finds a natural niche of application in the field
    of cryptanalysis. At the same time, neural networks offer a new approach to attack ciphering
    algorithms based on the principle that any function could be reproduced by a neural network,
    which is a powerful proven computational tool that can be used to find the inverse-function of
    any cryptographic algorithm.

    In this paper we examine the efficacy, feasibility and 
    general practicality of the use of Adversarial Neural Cryptography, 
    as coined by Abadi et al. in \cite{seminalanc}, and neural cryptography
    in general. We test the recommended systems and examine their use in 
    data transmission systems for the purpose of encrypting data, and also test
    them against the standard algorithms in for this purpose today.
  \end{abstract}
 
  \section{Introduction}
  
  \section{Related Work}
  
  \section{Testing Methodology}
  \lipsum
  
  \section{Results}
    \subsection{ANC}
    \subsection{Cryptonet}
  
  \section{Conclusions}

  \bibliographystyle{IEEEtran}
  \bibliography{../../resources/citations}
\end{document}
